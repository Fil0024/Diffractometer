\documentclass[a4paper,12pt]{article}
\usepackage[utf8]{inputenc}
\usepackage[T1]{fontenc}
\usepackage[polish]{babel}
\usepackage{geometry}
\usepackage{graphicx}
\usepackage{amsmath}
\usepackage{float}
\usepackage{caption}
\usepackage{subcaption}
\usepackage{booktabs}
\usepackage{hyperref}

\geometry{margin=2.5cm}

\title{\textbf{Raport z pomiarów dyfrakcji rentgenowskiej (XRD)}\\
Analiza stałej sieci i składu chemicznego próbek półprzewodnikowych}
\author{Imię i Nazwisko}
\date{\today}

\begin{document}

\maketitle

\section{Wstęp teoretyczny}

\subsection{Promieniowanie rentgenowskie i struktura krystaliczna}
Promieniowanie rentgenowskie (promieniowanie X) to rodzaj promieniowania elektromagnetycznego o długości fali w zakresie od 0,01 do 10 nm. W krystalografii najczęściej wykorzystuje się promieniowanie o długości fali rzędu angstremów (\AA), co odpowiada typowym odległościom międzypłaszczyznowym w ciałach stałych. W niniejszym eksperymencie wykorzystano linię emisyjną miedzi $Cu_{K\alpha 1}$ o długości fali:
\begin{equation}
    \lambda = 1.54056 \, \text{\AA}
\end{equation}
Badane próbki to kryształy o strukturze blendy cynkowej (typowa dla GaAs), która składa się z dwóch przesuniętych względem siebie sieci regularnych ściennie centrowanych (FCC). W analizie skupiono się na refleksie od płaszczyzn krystalograficznych o wskaźnikach Millera $(hkl) = (1, 1, 1)$. Odległość międzypłaszczyznowa $d_{hkl}$ dla układu regularnego wyraża się wzorem:
\begin{equation}
    d_{hkl} = \frac{a}{\sqrt{h^2 + k^2 + l^2}}
\end{equation}
gdzie $a$ to stała sieci krystalicznej.

\subsection{Prawo Bragga}
Zjawisko dyfrakcji na sieci krystalicznej opisuje prawo Wulffa-Bragga. Warunkiem koniecznym do wystąpienia konstruktywnej interferencji fal odbitych od równoległych płaszczyzn sieciowych jest spełnienie równania:
\begin{equation}
    2d_{hkl} \sin(\theta) = n\lambda
    \label{eq:bragg}
\end{equation}
gdzie $\theta$ to kąt padania wiązki (kąt Bragga), a $n$ to rząd dyfrakcji (w tym przypadku $n=1$). Prawo to pozwala na wyznaczenie stałej sieci $a$ na podstawie zmierzonego kąta $2\theta$.

\subsection{Domieszkowanie i prawo Vegarda}
Domieszkowanie polega na wprowadzeniu atomów innego pierwiastka do sieci krystalicznej materiału bazowego, co prowadzi do zmiany jego właściwości fizycznych oraz parametrów geometrycznych sieci. W przypadku roztworów stałych typu $A_{1-x}B_x$, stała sieci zmienia się liniowo wraz ze zmianą składu $x$. Zależność tę opisuje prawo Vegarda:
\begin{equation}
    a_{mix} = (1-x)a_{A} + x a_{B}
    \label{eq:vegard}
\end{equation}
gdzie $a_{mix}$ to zmierzona stała sieci stopu, $a_A$ to stała sieci materiału podłoża (np. GaAs), a $a_B$ to stała sieci materiału domieszki (np. InAs). Przekształcając ten wzór, można wyznaczyć zawartość domieszki $x$:
\begin{equation}
    x = \frac{a_{mix} - a_{A}}{a_{B} - a_{A}}
\end{equation}

\subsection{Budowa dyfraktometru i metody pomiarowe}
Dyfraktometr rentgenowski składa się z trzech głównych elementów: lampy rentgenowskiej (źródła), goniometru (uchwytu próbki) oraz detektora.
Wyróżniamy kilka podstawowych typów skanów:
\begin{itemize}
    \item \textbf{Skan $\omega$ (Rocking curve):} Detektor jest nieruchomy w pozycji piku dyfrakcyjnego, a próbka obraca się wokół osi $\omega$. Służy do oceny jakości krystalicznej (mozaikowości) warstwy.
    \item \textbf{Skan $2\theta$:} Próbka jest nieruchoma, a detektor porusza się po łuku.
    \item \textbf{Skan sprzężony $2\theta/\omega$:} Próbka obraca się o kąt $\theta$, a detektor jednocześnie o $2\theta$. Pozwala to na dokładne wyznaczenie odległości międzypłaszczyznowych.
\end{itemize}

Układ detekcyjny może być skonfigurowany w różnych wariantach wpływających na rozdzielczość i intensywność:
\begin{enumerate}
    \item \textbf{Bez analizatora (Open detector):} Najwyższa intensywność, niska rozdzielczość kątowa.
    \item \textbf{Szczelina przed detektorem:} Kompromis między intensywnością a rozdzielczością.
    \item \textbf{Analizator (kryształ):} Bardzo wysoka rozdzielczość, ale znaczny spadek intensywności (wymaga długich czasów zliczania).
\end{enumerate}

\subsection{Procedura ustawiania osi Z}
Kluczowym elementem kalibracji próbki jest ustawienie jej wysokości (oś Z). Procedura ta polega na ustawieniu goniometru w pozycji bezpośredniej wiązki ($\omega = 0, 2\theta = 0$). Następnie przesuwa się próbkę w osi Z, monitorując natężenie sygnału na detektorze. Prawidłowa pozycja to taka, w której próbka przesłania dokładnie połowę wiązki pierwotnej (natężenie spada do 50\% wartości maksymalnej). Zapewnia to, że oś obrotu goniometru leży dokładnie na powierzchni próbki.

\section{Metodyka analizy danych i niepewności}

Do analizy pików dyfrakcyjnych zastosowano dopasowanie funkcją Gaussa wraz z liniowym tłem. Na podstawie parametrów dopasowania wyznaczono środek piku ($2\theta_{meas}$) oraz jego szerokość połówkową (FWHM).

\subsection{Kalibracja instrumentalna}
Ze względu na stwierdzone wady techniczne układu pomiarowego, konieczne było wyznaczenie poprawki instrumentalnej (przesunięcia zera). Wykorzystano do tego próbkę referencyjną nr 1 (czyste podłoże GaAs).
Wartość teoretyczna kąta $2\theta$ dla GaAs (111) wynosi:
\begin{equation}
    2\theta_{ref} = 2 \arcsin\left( \frac{\lambda \sqrt{3}}{2 a_{GaAs}} \right)
\end{equation}
Przyjmując $a_{GaAs} = 5.653$ \AA, obliczono $2\theta_{theory} \approx 27.30^\circ$.
Poprawka $\Delta$ została zdefiniowana jako:
\begin{equation}
    \Delta = 2\theta_{measured, \#1} - 2\theta_{theory}
\end{equation}
Poprawka ta została odjęta od wyników pomiarowych dla wszystkich próbek.

\subsection{Szacowanie niepewności}
Przyjęto konserwatywne założenie, że niepewność wyznaczenia pozycji piku $u(2\theta)$ jest powiązana z jego szerokością połówkową (FWHM). Do obliczeń wykorzystano metodę propagacji niepewności (różniczka zupełna).

Niepewność stałej sieci $u(a)$:
Ze wzoru $a = \frac{\lambda \sqrt{3}}{2 \sin\theta}$ (dla $2\theta/2 = \theta$):
\begin{equation}
    u(a) = \left| \frac{da}{d\theta} \right| u(\theta) = a \cdot \cot(\theta) \cdot u(\theta_{rad})
\end{equation}
gdzie $u(\theta_{rad})$ to niepewność kąta $\theta$ wyrażona w radianach (połowa niepewności $2\theta$).

Niepewność składu $u(x)$:
Korzystając z prawa Vegarda:
\begin{equation}
    u(x) = \frac{u(a)}{|a_{B} - a_{A}|}
\end{equation}

\section{Wyniki eksperymentu}

\subsection{Próbka \#1 (Referencyjna GaAs)}
Dla próbki nr 1 wykonano skan $2\theta/\omega$. Wynik dopasowania przedstawiono na Rysunku \ref{fig:sample1_fit}.

\begin{figure}[H]
    \centering
    \includegraphics[width=0.7\textwidth]{results/plot_#1_fit_lin.pdf}
    \caption{Dopasowanie krzywej Gaussa do refleksu (111) dla Próbki \#1 (skala liniowa).}
    \label{fig:sample1_fit}
\end{figure}

Wyznaczony kąt $2\theta$ posłużył do kalibracji. Wykres krzywej kołysania (omega scan) pokazuje symetryczny kształt (Rys. \ref{fig:sample1_omega}), co świadczy o wysokiej jakości krystalicznej podłoża.

\begin{figure}[H]
    \centering
    \includegraphics[width=0.7\textwidth]{results/plot_#1_omega.pdf}
    \caption{Skan omega (Rocking Curve) dla Próbki \#1.}
    \label{fig:sample1_omega}
\end{figure}

\subsection{Próbka \#2 (Słabo domieszkowana)}
Po uwzględnieniu poprawki kalibracyjnej, wyznaczono stałą sieci oraz skład $x$.
Wynik dopasowania $2\theta/\omega$ widoczny jest na Rysunku \ref{fig:sample2_fit}.

\begin{figure}[H]
    \centering
    \includegraphics[width=0.7\textwidth]{results/plot_#2_fit_lin.pdf}
    \caption{Dopasowanie krzywej Gaussa dla Próbki \#2.}
    \label{fig:sample2_fit}
\end{figure}

Analiza skanu omega (Rys. \ref{fig:sample2_omega}) wykazuje, że pik jest nieco szerszy niż dla próbki referencyjnej. Można zauważyć subtelne zaburzenie kształtu Gaussa (delikatna asymetria), co sugeruje początek zmian w strukturze wywołanych domieszkowaniem.

\begin{figure}[H]
    \centering
    \includegraphics[width=0.7\textwidth]{results/plot_#2_omega.pdf}
    \caption{Skan omega dla Próbki \#2.}
    \label{fig:sample2_omega}
\end{figure}

\textbf{Wyniki obliczeń dla Próbki \#2:}
\begin{itemize}
    \item Zmierzony kąt $2\theta$ (skorygowany): 27.2435$^\circ$
    \item Stała sieci $a$: 5.665 \AA
    \item Wyliczony skład $x$: 0.029 $\pm$ 0.016
\end{itemize}

\subsection{Próbka \#3 (Mocno domieszkowana)}
W przypadku próbki nr 3 przesunięcie piku w stronę mniejszych kątów jest wyraźne, co oznacza wzrost stałej sieci.

\begin{figure}[H]
    \centering
    \includegraphics[width=0.7\textwidth]{results/plot_#3_fit_lin.pdf}
    \caption{Dopasowanie krzywej Gaussa dla Próbki \#3.}
    \label{fig:sample3_fit}
\end{figure}

Skan omega dla tej próbki (Rys. \ref{fig:sample3_omega}) różni się drastycznie od referencyjnego. Widać wyraźne rozdwojenie piku lub silną asymetrię – dwa nałożone na siebie rozkłady (jeden silniejszy, drugi słabszy). Świadczy to o segregacji domieszki, istnieniu bloków mozaikowych o nieco różnej orientacji lub gradientach naprężeń w silnie domieszkowanej warstwie.

\begin{figure}[H]
    \centering
    \includegraphics[width=0.7\textwidth]{results/plot_#3_omega.pdf}
    \caption{Skan omega dla Próbki \#3 – widoczna struktura wielopikowa.}
    \label{fig:sample3_omega}
\end{figure}

\textbf{Wyniki obliczeń dla Próbki \#3:}
\begin{itemize}
    \item Zmierzony kąt $2\theta$ (skorygowany): 27.1235$^\circ$
    \item Stała sieci $a$: 5.689 \AA
    \item Wyliczony skład $x$: 0.089 $\pm$ 0.027
\end{itemize}

\section{Zestawienie zbiorcze i weryfikacja}

Poniżej przedstawiono tabelę zbiorczą uzyskanych wyników (Tabela \ref{tab:results}). Wartości $x$ zostały wyznaczone przy założeniu, że drugim składnikiem jest InAs ($a_B \approx 6.0583$ \AA).

\begin{table}[H]
\centering
\caption{Wyniki analizy dyfrakcyjnej dla badanych próbek.}
\label{tab:results}
\begin{tabular}{@{}lcccc@{}}
\toprule
Próbka & $2\theta_{corr}$ [$^\circ$] & FWHM [$^\circ$] & Stała sieci $a$ [\AA] & Skład $x$ (Vegard) \\ \midrule
\#1 (Ref) & 27.3032 & 0.046 & 5.653 & 0.000 \\
\#2       & 27.2435 & 0.065 & 5.665 & 0.029 $\pm$ 0.016 \\
\#3       & 27.1235 & 0.111 & 5.689 & 0.089 $\pm$ 0.027 \\ \bottomrule
\end{tabular}
\end{table}

\begin{figure}[H]
    \centering
    \includegraphics[width=0.8\textwidth]{results/plot_vegard_law.pdf}
    \caption{Wizualizacja punktów pomiarowych na tle teoretycznej linii prawa Vegarda.}
    \label{fig:vegard}
\end{figure}

\section{Wnioski}
Przeprowadzone badania dyfrakcyjne pozwoliły na wyznaczenie składu chemicznego trójskładnikowych roztworów stałych.
\begin{enumerate}
    \item Zastosowanie poprawki na przesunięcie zera goniometru było kluczowe dla uzyskania poprawnej wartości stałej sieci próbki referencyjnej GaAs.
    \item Próbka \#2 wykazuje niewielkie domieszkowanie ($x \approx 3\%$). Jej jakość krystaliczna jest dobra, choć widoczne jest niewielkie poszerzenie krzywej kołysania.
    \item Próbka \#3 posiada znacznie wyższą zawartość domieszki ($x \approx 9\%$). Analiza skanu omega ujawniła znaczną degradację jakości strukturalnej (asymetria/rozdwojenie piku), co jest typowym efektem relaksacji naprężeń lub niejednorodności składu przy wyższym poziomie domieszkowania.
    \item Prawo Vegarda umożliwiło powiązanie zmian kąta dyfrakcji ze zmianą składu chemicznego stopu.
\end{enumerate}

\end{document}