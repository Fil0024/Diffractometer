\documentclass[a4paper,12pt]{article}
\usepackage[utf8]{inputenc}
\usepackage[T1]{fontenc}
\usepackage[polish]{babel}
\usepackage{geometry}
\usepackage{graphicx}
\usepackage{amsmath}
\usepackage{float}
\usepackage{caption}
\usepackage{subcaption}
\usepackage{booktabs}
\usepackage{hyperref}
\usepackage{icomma}
\usepackage{subcaption}
\usepackage{placeins}
\usepackage{siunitx}

% Podwójne klamry są tutaj wymagane!
\graphicspath{{../results/}}

\geometry{margin=2.5cm}

\title{\textbf{X3 – POMIAR STAŁYCH SIECI PÓŁPRZEWODNIKOWYCH ROZTWORÓW
STAŁYCH METODĄ DYFRAKTOMETRII RENTGENOWSKIEJ – PRAWO
VEGARDA}}
\author{Filip Gołębiewski}
\date{\today}

\begin{document}

\maketitle

\begin{abstract}
W doświadczeniu wykonano pomiary z wykorzystaniem dyfraktometru na trzech próbkach. Pierwsza próbka była wykonano z czystego GaAs, a pozostałe dwie były domieszkowane InAs. Z wykorzystaniem zjawiska dyfrakcji rentgenowskiej zmierzono wartości kąta Bragga dla każdej z próbek i obliczono wartości: stałej sieci oraz domieszkowania indem próbek domieszkowanych.

W przypadku próbki słabiej domieszkowanej otrzymano wartość stałej sieci wynoszącą $a = 5,662 +- 0,003$ \AA{} oraz procentową zawartość indu wartości $2,2 +- 0,7$~\%, natomiast dla próbki bardziej domieszkowanej otrzymano stałą sieci $5,691 +- 0,004$ \AA{} i procentową zawartość indu $9 +- 1$~\%.
\end{abstract}

\section{Wstęp teoretyczny}

\subsection{Promieniowanie rentgenowskie}
W eksperymencie wykorzystano promieniowanie rentgenowskie, ponieważ w celu uzyskania dyfrakcji na warstwach krystalicznych konieczne jest wykorzystanie promieniowania o długości fali o tym samym rzędzie wielkości co odległość międzypłaszczyznowa, czyli w przybliżeniu 1 \AA{}. Źródło promieniowania w wykorzystane w niniejszym dokumencie miało anodę miedzianą, wykorzystano monochromator, aby odseparować tylko jedną linię emisyjną $Cu_{K\alpha 1}$ o długości fali:
\begin{equation}
\lambda = 1.54056 \, \text{\AA}
\end{equation}

\subsection{Struktura krystaliczna}
Badane próbki to kryształy: GaAs oraz dwie próbki GaAs o różnym stopniu domieszkowania InAs. W analizie wykorzystano refleks o wskaźnikach Millera $(hkl)=(1,1,1)$ w układzie regularnym.

Odległość międzypłaszczyznowa $d_{hkl}$ dla układu regularnego wyraża się wzorem:
\begin{equation}
d_{hkl} = \frac{a}{\sqrt{h^2 + k^2 + l^2}}
\label{eq:odleglosc_d}
\end{equation}
gdzie $a$ to stała sieci krystalicznej.

\subsection{Prawo Bragga}
Zjawisko dyfrakcji na sieci krystalicznej opisuje prawo Bragga. Warunkiem koniecznym do wystąpienia konstruktywnej interferencji fal odbitych od równoległych płaszczyzn sieciowych jest spełnienie równania:
\begin{equation}
n\lambda = 2d_{hkl} \sin(\theta)
\label{eq:bragg}
\end{equation}
gdzie $\theta$ to kąt padania wiązki (kąt Bragga), a $n$ to rząd dyfrakcji (w tym przypadku $n=1$). Prawo to pozwala na wyznaczenie stałej sieci $a$ na podstawie zmierzonego kąta $2\theta$.

\subsection{Prawo Vegarda}
W niniejszym eksperymencie wykorzystano prawo Vegarda do obliczenia składu w stopie trójskładnikowym In$_x$Ga$_{1-x}$As. W takim przypadku prawo Vagarda można wyrazić jako:
\begin{equation}
a=a_{GaAs}(1-x)+a_{InAs}
\label{eq:vegard}
\end{equation}
gdzie x to proporcjonalna zawartość In, a $1- x$ to
zawartość Ga, natomiast $a$ to stała sieci stopu.

\section{Układ pomiarowy}

\subsection{Budowa dyfraktometru i wykorzystane typy skanów}
Dyfraktometr rentgenowski składa się z trzech głównych elementów: lampy rentgenowskiej, goniometru oraz detektora.
Wyróżniamy kilka podstawowych typów skanów:
\begin{itemize}
\item \textbf{Skan $\omega$:} Detektor jest nieruchomy w pozycji piku dyfrakcyjnego, a próbka obraca się wokół osi $\omega$. Służy do oceny jakości krystalicznej warstwy.
\item \textbf{Skan $2\theta$:} Próbka jest nieruchoma, a detektor porusza się po łuku.
\item \textbf{Skan sprzężony $2\theta/\omega$:} Próbka obraca się o kąt $\theta$, a detektor jednocześnie o $2\theta$.
\end{itemize}
W celu ustawienia próbki i detektora w odpowiednich pozycjach do zarejestrowania dyfrakcji wykonywano kilka kalibracyjnych skanów $\omega$ i $2\theta$ na zmianę. Skany zawarte w tym raporcie wykonano po dokładnej kalibracji z ustawionym najmnijszym dostępnym krokiem (0,005 $^\circ$).

Do pomiarów wykorzystano dyfraktometr Philips MRD~\cite{dyfraktometr}.

\subsection{Dostępne konfigurację detektora}

Wykorzystywano trzy konfiguracje detektora różniące się rozdzielczością oraz intensywnością rejestrowanego sygnału. Skalibrowano urządzenie tak, aby przy po ustawieniu kąta $2/theta$ wiązka ze źródła wpadała do detektora z analizatorem, w celu przełączenia się na detektor bez analizatora zgodnie z instrukcją dyfraktometru \cite{dyfraktometr} dodawano do kąta $2\theta$ $4,55^\circ$, ostatnia konfiguracja również wykorzystuje detektor bez analizatora, ale ze wsuniętą płytką z wąską szczeliną.

W celu sprawdzenia charakterystyki konfiguracji detektora wykonano trzy skany $2\theta$ (Rys.~\ref{fig:calib}). Widać, że konfiguracja bez analizatora daje najmniejszą rozdzielczość, ale największą intensywność, natomiast konfiguracja z analizatorem daje najlepszą rozdzielczość kosztem intensywności, konfiguracja ze szczeliną daje wartości pośrednie między nimi.

\begin{figure}[ht]
\centering
\includegraphics[width=0.9\textwidth]{plot_calib_combined_shifted.pdf}
\caption{Porównanie skanów $2\theta$ wykonanych każdym z detektorów. Detektor bez analizatora (pomarańczowy), ze szczeliną (zielony) oraz z analizatorem (niebieski). Widoczne przy osiach jednostki to: deg, czyli stopnie oraz cps, czyli zliczenia na sekundę (eng. counts per second)}\label{fig:calib}
\end{figure}

\subsection{Ustawianie osi Z}
W celu uzyskania poprawnych wyników, konieczne było odpowiednie ustawienie osi z, czyli dostosowanie aparatury do grubości konkretnej próbki. W celu kalibracji osi z ustawiano kąty $\omega$ oraz $2\theta$ na zero i przesuwano próbkę do momentu, aż będzie ona blokowało około połowę wiązki. Taka procedura ustawiania osi z jest konieczna, aby po ustawieniu kąta omega cała wiązka padała na powierzchnię próbki.

\section{Niepewności pomiarowe}
Jako niepewność pomiarową $2\theta$ przyjęto szerokość połówkową (FWHM) rozkładu Gaussa dopasowanego do wykresu natężenia padającej wiązki. Jako FWHM przyjęto:
\begin{equation}
FWHM = 2,335 \cdot \sigma
\end{equation}
gdzie $\sigma$ to odchylenie standardowe otrzymane przy dopasowaniu rozkładu Gaussa.

Pozostałe niepewności obliczano z wykorzystaniem wzoru na propagacje niepewności:
\begin{equation}
u_f=\sqrt{ \sum_{i=1}^n \left( \frac{\partial f}{\partial x_i}\cdot u_{x_i} \right)^2 }
\end{equation}

\newpage
\section{Analiza danych}

Ze względu na wady techniczne układu pomiarowego, konieczne było wyznaczenie poprawki. Wykorzystano do tego próbkę referencyjną (czysty GaAs).

Wartość teoretyczna~\eqref{eq:bragg} kąta $2\theta$ dla GaAs (111) wynosi:
\begin{equation}
2\theta_{teoretyczne} = 2 \arcsin\left( \frac{\lambda \sqrt{3}}{2 a_{GaAs}} \right)
\end{equation}
Przyjmując $a_{GaAs} = 5.653$ \AA, obliczono $2\theta_{teoretyczne} \approx 27.30^\circ$.
Poprawka $\Delta$ została zdefiniowana jako:
\begin{equation}
\Delta = 2\theta_{zmierzone} - 2\theta_{teoretyczne}
\end{equation}
Poprawka ta została odjęta od wyników pomiarowych dla wszystkich próbek.

\subsection{Analiza skanów $2\theta$/$\omega$}

Dla każdej z próbek wykonano skan $2\theta$/$\omega$ i dopasowano do nich rozkład Gaussa:
\begin{equation}
f(x) = A \cdot \exp \left(-\frac{\left( x - B \right)^2}{2 C^2}\right)+Dx+E
\end{equation}
gdzie $A$ jest stałą określającą maksymalną wartość krzywej Gaussa, $B$ jest średnią wartością rozkładu Gaussa, czyli $B=2\theta$, $C=\sigma$ to odchylenie standardowe, a $D$ i $E$ to współczynniki zależności liniowej, która została dodana, aby uwzględnić szum (Rys. \ref{fig:sample1_fit}, \ref{fig:sample2_fit}, \ref{fig:sample3_fit}).

\begin{table}[ht]
\centering
\caption{Wyniki analizy dyfrakcyjnej dla badanych próbek.}\label{tab:dopasowania}
\begin{tabular}{@{}lccc@{}}
\toprule
Parametr & Próbka 1 & Próbka 2 & Próbka 3 \\ \midrule
$A$ (amplituda) [deg] & $\num{14980 +- 40}$ & $\num{3385 +- 8}$ & $\num{858 +- 4}$ \\ \addlinespace
$B$ (średnia) [deg] & $\num{27,37091 +- 0,00001}$ & $\num{27,32787 +- 0,00001}$ & $\num{27,18220 +- 0,00004}$ \\ \addlinespace
$C$ ($\sigma$) [deg] & $\num{0,004002 +- 0,000012}$ & $\num{0,004038 +- 0,000011}$ & $\num{0,006995 +- 0,000038}$ \\ \addlinespace
$D$ (nachylenie) [cps/deg] & $\num{-3 +- 36}$ & $\num{4 +- 5}$ & $\num{2,5 +- 2,7}$ \\ \addlinespace
$E$ (tło) [cps] & $\num{111 +- 995}$ & $\num{-113 +- 149}$ & $\num{71 +- 75}$ \\ \bottomrule
\end{tabular}
\end{table}

Z wykorzystaniem wzorów

Dla próbki referencyjnej (czysty GaAs), otrzymano wartość $A=2\theta_{zmierzone}=27,3709^\circ$, obliczono poprawkę $\Delta = 0.0686^\circ$, którą uwzględniono we wszystkich wynikach.

Wartości stałej sieci $a$ (wzory~\eqref{eq:odleglosc_d} oraz~\ref{eq:bragg}) oraz stężenie InAs $x$ (wzór~\eqref{eq:vegard}) obliczono, wykorzystując wzory teoretyczne. Jako niepewność średnij wartości kąta (środek krzywej Gaussa) przyjęto FWHM, natomiast niepewności $a$ oraz $x$ obliczono z wykorzystaniem propagacji niepewności:
\begin{equation}
u_a = \sqrt{\left( \frac{\partial a}{\partial (2\theta)} \cdot u_{2\theta} \right)^2} = \frac{a}{2} \cot(\theta) \cdot u_{2\theta}^{rad}
\end{equation}
\begin{equation}
u_x = \left| \frac{\partial x}{\partial a} \right| u_a = \frac{u_a}{|a_{InAs} - a_{GaAs}|}
\end{equation}
Wyniki dla wszystkich próbek przedstawiono w Tabeli~\ref{tab:results}.

\begin{table}[ht]
\centering
\caption{Wyniki analizy dyfrakcyjnej. Wartości stałej sieci $a$ oraz stężenie InAs $x$.}\label{tab:results}
\begin{tabular}{@{}lccc@{}}
\toprule
& Próbka 1 & Próbka 2 & Próbka 3 \\ \midrule
$2\theta$ [deg] & 27,30$\pm$0,01 & 27,26$\pm$0,01 & 27,11$\pm$0,02 \\ \addlinespace
$\theta_B$ [deg] & 13,651 $\pm$ 0,005& 13,630 $\pm$ 0,005 & 13,557 $\pm$ 0,008 \\ \addlinespace
$a$ [\AA] & 5,653 $\pm$ 0,002 & 5,662 $\pm$ 0,002 & 5,692 $\pm$ 0,003 \\ \addlinespace
$x$ [\%] & 0 & 2,2 $\pm$ 0,5 & 9,5 $\pm$ 0,8 \\ \addlinespace
\\ \bottomrule
\end{tabular}
\end{table}

\begin{figure}[ht]
\centering
\includegraphics[width=0.9\textwidth]{plot_sample1_fit_lin.pdf}
\caption{Dopasowanie krzywej Gaussa do skanu $2\theta$/$\omega$ dla Próbki 1 (referencyjna).}\label{fig:sample1_fit}
\end{figure}

\begin{figure}[ht]
\centering
\includegraphics[width=0.9\textwidth]{plot_sample2_fit_lin.pdf}
\caption{Dopasowanie krzywej Gaussa do skanu $2\theta$/$\omega$ dla Próbki 2.}\label{fig:sample2_fit}
\end{figure}

\begin{figure}[ht]
\centering
\includegraphics[width=0.9\textwidth]{plot_sample3_fit_lin.pdf}
\caption{Dopasowanie krzywej Gaussa do skanu $2\theta$/$\omega$ dla Próbki 3.}\label{fig:sample3_fit}
\end{figure}

\FloatBarrier\subsection{Analiza jakościowa skanów $\omega$}

Wszystkie trzy próbki zbadano także z wykorzystaniem skanu omega. Najpierw ustawiono detektor w położeniu dogodnym do rejestrowania dyfrakcji ($2\theta_B$), następnie wykonano skan $\omega$, czyli przekręcano próbkę względem wiązki rentgenowskiej. Skan ten pozwolił na analizę jakościową powierzchni próbki.

Na skanie próbki pierwszej (Rys. \ref{fig:sample1_omega}), zarejestrowano wyraźny pik bez poważnych zniekształceń. Pozwala to stwierdzić wysoką jakość sieci krystalicznej tej próbki.
W przypadku próbki drugiej (Rys. \ref{fig:sample2_omega}) zarejestrowano zniekształcenie piku, które przypomina drugi pik blisko głównego. Oznacza to gorszą jakość sieci krystalicznej oraz obecność defektów i przesunięć w strukturze krystalicznej. Można zauważyć także, że pik ten jest znacznie szerszy od pierwszego (około 0,05$^\circ$ przy podstawie, a w pierwszej próbce około 0,01$^\circ$).
Najgorszej jakości okazała się próbka trzecia (Rys. \ref{fig:sample3_omega}). W tym przypadku również widać dwa zbliżone piki, ponadto szerokość piku jeszcze bardziej się powiększyła (do około 0.2$^\circ$).

Wskazuje to, że próbki domieszkowane charakteryzują się gorszą jakością krystaliczną w porównaniu do wzorca. Obserwowany efekt, w tym wielopikowa struktura widoczna szczególnie wyraźnie na Rysunku~\ref{fig:sample3_omega}, może wynikać z obecności defektów sieciowych (np. dyslokacji) wprowadzonych w procesie domieszkowania, naprężeń w sieci krystalicznej lub istnienia bloków mozaikowych (podziaren) o nieznacznie różnej orientacji krystalograficznej.
\begin{figure}[!htbp]
\centering
\includegraphics[width=0.9\textwidth]{plot_sample1_omega.pdf}
\caption{Skan omega (Rocking Curve) dla Próbki \#1.}\label{fig:sample1_omega}
\end{figure}

\begin{figure}[!htbp]
\centering
\includegraphics[width=0.9\textwidth]{plot_sample2_omega.pdf}
\caption{Skan omega dla Próbki \#2.}\label{fig:sample2_omega}
\end{figure}

\begin{figure}[!htbp]
\centering
\includegraphics[width=0.9\textwidth]{plot_sample3_omega.pdf}
\caption{Skan omega dla Próbki 3, widoczna struktura wielopikowa.}\label{fig:sample3_omega}
\end{figure}

\FloatBarrier
\section{Podsumowanie}
W ramach przeprowadzonego eksperymentu zrealizowano pomiary dyfraktometryczne mające na celu wyznaczenie stałych sieci oraz składu chemicznego półprzewodników. Wykorzystanie prawa Bragga pozwoliło określić kąty dyfrakcji. Na podstawie obliczonych stałych sieci i zastosowaniu prawa Vegarda wyznaczono stężenie InAs w badanych próbkach, potwierdzając, że wprowadzenie atomów indu do sieci GaAs powoduje wzrost stałej sieci, co jest zgodne z przewidywaniami teoretycznymi. Analiza jakościowa profilu refleksów dyfrakcyjnych wykazała, że proces domieszkowania wpływa na strukturę rzeczywistą materiału, prowadząc do obniżenia jego doskonałości krystalicznej w porównaniu do czystego podłoża.

\begin{thebibliography}{9}

\bibitem{dyfraktometr}
Philips, \emph{MRD – User guide}, Instrukcja obsługi do dyfraktometru Philips MRD, Philips, 1993.

\end{thebibliography}

\end{document}


